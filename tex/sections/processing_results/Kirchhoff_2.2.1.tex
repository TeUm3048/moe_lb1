\subsection{Исследование цепи при питании её от двух источников }

Проверим результаты эксперимента, используя уравнения Кирхгофа.
Для этого составим систему уравнений:

\begin{equation}
  \begin{cases}
    I_1 + I_2 - I_3 = 0 \\
    I - I_2 - I_4 = 0   \\
    U_1 + U_3 - U = 0   \\
  \end{cases}
\end{equation}

Подставим в уравнения значения напряжений и токов:
\begin{equation}
  \begin{cases}
    I_1 + I_2 - I_3 = 0,206 + 0,274 - 0,475 = 0,005\ (\text{мА}) \\ % OK
    I - I_2 - I_4   = 0,989 - 0,274 - 0,680 = 0,035\ (\text{мА})   \\ % Тут говно
    U_1 + U_3 - U   = 0,31 + 1,57 - 1,92    = 0,04 \ (\text{мА})       \\ % OK
  \end{cases}
\end{equation}

Учитывая погрешность измерений, можно сделать вывод, что результаты
эксперимента совпадают с результатами, полученными с помощью
уравнений Кирхгофа.


