\subsection{Определение токов методом наложения}
Токи в цепи можно определить методом наложения, а именно:
ток в цепи равен алгебраической сумме токов, которые протекали бы
в цепи при включении каждого источника по отдельности.

Чтобы определить это составим таблицу, в которой будут указаны токи:

\begin{tabular}{|c|c|c|c|c|}
    \hline
    Включены источники & $I_1$, мА & $I_2$, мА & $I_3$, мА & $I_4$, мА \\
    \hline
    $U$                & $0,528$   & $-0,238$  & $0,301$   & $0,245$   \\
    \hline
    $I $               & $-0,358$  & $0,534$   & $0,138$   & $0,475$   \\
    \hline
    $U, I$             & $0,170$   & $0,296$   & $0,475$   & $0,720$   \\
    \hline
    $U, I$ (измерено)  & $0,206$   & $0,274$   & $0,439$   & $0,680$   \\
    \hline
\end{tabular}

% Пук-пук плак-плак, результаты не совпадают