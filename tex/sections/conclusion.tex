\section{Выводы}

В ходе выполнения лабораторной работы были изучены основные методы
анализа линейных электрических цепей: метод наложения, 
метод эквивалентного источника и принцип взаимности. 

Проведенные измерения подтвердили правильность применения уравнений Кирхгофа для анализа электрической цепи, принцип суперпозиции, принцип эквивалентного источника и принцип взаимности.

Метод наложения позволил успешно определить токи в различных ветвях цепи. Сравнение экспериментальных значений с теоретическими расчетами показало, что результаты совпадают в пределах допустимой погрешности, что подтверждает эффективность данного метода в анализе электрических цепей.

Использование эквивалентного источника напряжения для расчета тока в ветви с сопротивлением R3 также дало удовлетворительные результаты. Полученное значение тока (0,533 мА) близко к экспериментально измеренному (0,491 мА), что подтверждает правильность выбранного подхода.

Экспериментальная проверка принципа взаимности также подтвердила его справедливость: токи в разных ветвях при переносе источника напряжения остались одинаковыми (0,32 мА), что соответствует теоретическим ожиданиям.
